\documentclass[sigconf,nonacm]{acmart}

%% packages
\usepackage{graphicx}
\usepackage{lipsum}

\title{Arcadia Finance [Draft]}
\subtitle{November 2023}
\date{November 2023}

\author{Thomas Smets}
\affiliation{
    \institution{Arcadia Finance}
    \city{Brussels}
    \country{Belgium}
}
\email{thomas@arcadia.finance}

\renewcommand{\shortauthors}{Smets et al.}

%% Add whitespace
\begin{teaserfigure}
    \vspace{1em}
    \Description{Just used to add a whitespace over the length of the document.}
\end{teaserfigure}

\begin{document}

\begin{abstract}
    (To Do)
    Arcadia Finance introduces a novel on-chain architecture for collateral management of margined positions.

    The overarching vision of Arcadia Finance is to build standardized on-chain infrastructure for collateral management,
    unlocking the capital in diverse DeFi assets, and establishing a comprehensive platform for managing collateralized positions. 

    As we anticipate the continued growth of on-chain tokenized assets and financial primitives, the importance of standardized and efficient collateral management becomes increasingly paramount.
    Without it, the full potential of these assets and financial primitives remains unrealized, hindered by protocol inflexibility and fragmentation.

    Arcadia strives to enable higher capital efficiency for users, transparent risk assessment, faster go-to-market time for new protocols without compromising on security, self-custody and decentralisation.
\end{abstract}

\keywords{Collateral management, DeFi, EVM}

\maketitle

\section{Introduction (To Do)} 
\label{sec:introduction}
This paper will explain the rationale why we are building the Arcadia Finance Technology Stack, and go over its technical implementation.

The Arcadia Financial Technology Stack is a set of on-chain, inter-linked financial protocols, centered around collateral management.
They facilitate non-trusting peers to close financial contracts without the need of intermediaries.

It consists out of three protocols, each with distinct responsibilities and permissions:
\begin{itemize}
\item The Arcadia Registry.
\item Arcadia Accounts.
\item A Creditor contract.
\end{itemize}

\section{Terminology}
\textsc{Debtors and Creditors:} We will often use the terms Debtors and Creditors throughout this paper,
our definition is broader than the commonly accepted context within traditional debt arrangements.
We use the term Debtor for any holder of a financial instrument that results in liabilities on the balance sheet 
(borrowers, option writers, payers of cash-settled futures contract or swaps...). 
While the Creditor refers to the entity to whom the liability is owed.

\textsc{Open position:} The size of the liability that a specific Debtor has with a specific Creditors.

\textsc{Collateral:} The assets pledged by a Debtor to a Creditor to cover the credit risk in case the Debtor would default.

\textsc{Margin:} The value of collateral assets that a Debtor must hold in a Margin Account to cover the credit risk of its Creditor(s).
The margin requirements, are set by the Creditor.
In general there are two types of margin requirements: Initial Margin (minimum value of collateral that must be held in the margin account to open a new position)
and Maintenance Margin (minimum value of collateral that must be held in the margin account to keep an existing position open).

\textsc{Margin Account:} Our definition of a Margin Account extends beyond the conventional usage within Brokerage Accounts. 
The holder of a Margin Account not only has the ability to utilize and transact with the assets within the account (as is the case with a cash account or your standard crypto wallet),
but can also incur liabilities against those assets. The margin requirements (initial and maintenance margin) are set by the Creditor.

\section{History of Collateral Management}
\label{subsec:history}

Withing the financial system, Collateral serves a primary function in mitigating the counterparty risks borne by Creditors, in case Debtors would fail to fulfill their financial liabilities.

\subsection{Origin}
Evidence of collateralized loans, also known as secured loans, can be traced back to at least 400 BCE in Ancient Greece  \cite{millett2002lending}.
In these early transactions, borrowers provided tangible assets as collateral to secure loans, laying the foundation for the concept of collateralization.

\subsection{Current State}
Over the past decades we have witnessed significant advancements in collateral management practices.

With the rise of complex financial instruments and the globalization of markets, the role of collateral has expanded beyond traditional lending.
Collateral is now used to secure a wide range of financial instruments, including derivatives trading, margin accounts, and other sophisticated financial contracts.

The digitalisation of the financial system greatly improved the efficiency of the exchange and settlement of collateral.
Reporting, margining and reconciliation processes could be executed on a daily basis instead of on a weekly or even monthly basis \cite{simmons2019collateral}.
% Add some figures or numbers here?

The emergence of blockchain technology and decentralized finance (DeFi) led to a new wave of innovation for collateral management.
Blockchain has some excellent properties with regards to collateral management:
\begin{itemize}
    \item With it's immutable and transparent ledger, collateral can be audited 24/7.
    \item Smart contracts enforce margin requirements real time on a 24/7 continuous basis.
    \item The atomic execution of transactions avoid expensive reconciliation processes.
    \item The atomic execution also enables optimistic execution of transactions (e.g. flash loans) which can greatly improve processes such as refinancing loans or even enable completely new financial use-cases.
    \item The permissionless nature of blockchain and Dapps (Decentralised Applications), combined with a shared state between all those applications, make different financial applications and assets composable and inter-operable by default.
\end{itemize}

It is no coincidence that some of the first blockchain applications with product market fit, were secured lending protocols.
These early pioneers like Maker\cite{team2017dai}, Aave\cite{thornburg2020aave} or Compound\cite{leshner2019compound} rely on collateralization as the only way to mitigate counter party risks and work with over-collateralized loans.
As such they enable peer-to-peer or even peer-to-contract loans, without the need of any trusted intermediaries.

\subsection{Collateral management in crisis}
Collateralization, both in the traditional financial world and in DeFi, is not without its problems.
Recent crises in both industries highlight the need for better collateral management infrastructure.

Bad collateral management, rehypothecation of collateral, opaque accounting practices of collateral,
and outright fraud with collateralized assets, are all cited as root causes\cite{hellwig2008causes} behind the 2008 financial crisis.
Following the crisis, a variety of major regulations were introduced (e.g. EMIR in the European Union, or Dodd-Frank in the USA).
While these packages were successful in stabilizing the financial system, they have clear centralization tendencies\cite{gregory2014central}.
The major beneficiaries of said regulations are the major established institutions, mandatory intermediaries for central clearing, central banks and the regulatory bodies themselves.
This might paradoxically cement even greater systemic risks into the financial system.

In 2022 DeFi experienced it's own financial crisis.
In a few months time, many of the ecosystem's key (albeit mostly centralised) players collapsed, and removed all liquid with them.
Again bad collateral management, rehypothecation of collateral, opaque protocol mechanisms, and outright fraud with collateralized assets were at the root of the problem.

A recurring trend with protocols that imploded in 2021 and 2022 is that the lack of collateralization was obfuscated behind complex protocol designs and narratives. 
Some of the notorious examples are:
\begin{itemize}
    \item Synthetic stable-tokens such as Gaia-USDf, Iron-Titan and Luna-Terra.
    \item Olympus and its forks relying on the (3,3) model.
    \item Protocols where the only utility of a token is to receive more of the token itself (animal yield farms, reflective tokens…).
    \item Unsecured loans by market makers and CeFi players.
\end{itemize}

DeFi protocols should focus more on collateralization, and less on Complex tokenomics that obfuscate who ends up paying when things go bad.

\subsection{The Future?}
The over-collateralized DeFi protocols however, operated remarkably well during the aforementioned crisis, even better than their traditional counterparts.
This is well illustrated by some of the collapsed centralised entities, active in the space until 2022, that had both on- and off-chain liabilities.
All on-chain Creditors were repaid in a timely and orderly fashion, as enforced by their smart contracts.
While the lawsuits for the many off-chain Creditors are still ongoing (and probably will be for many years).

It is for these types of crises that liabilities are secured in the first place.
These events further strengthened our vision that blockchain-based protocols have the potential to become the dominant infrastructure for collateral management.
But before we get there, a number of Inefficiencies of in current DeFi protocols must be overcome, as we will discuss in the next Section.

\section{Collateral Management in DeFi}
\subsection{General Principles}
% To Do, use picture, keep it high level

\subsection{Inefficiencies}

In the next sections we will highlight a number of inefficiencies,
which need to be overcome before DeFi can be broadly adopted as the go-to infrastructure for collateral management.

\subsubsection{Hacks and exploits}
\label{subsubsec:hacks-and-exploits}

The major problem in DeFi today is the number of hacks and exploits, the estimated amount of funds lost in 2023 ranges from \$400M to \$1B.
There are many security related practices the sector as a whole should improve on, but we want to highlight one problem.

Almost all DeFi protocols use over-collateralization in some way of form to manage counterparty risks between different user-groups.
Lending protocols, perpetual protocol, Prime brokerage protocols, option protocols etc. all require some or all users to deposit collateral.
While the nature of the financial contracts may be very different for each of these protocols, they share a great amount of common logic:
\begin{itemize}
    \item Pricing of collateralized assets.
    \item Management (depositing, withdrawing...) of collateralized assets.
    \item asset-liability specific risk parameters and calculations.
    \item Margin calls and liquidations of risky positions.
    \item Settling bad debt.
\end{itemize}

Pricing of assets, management of assets and liquidation logic is complex, error prone and most mistakes easily lead to severe user losses.
Today all protocols implement these redundantly, even for new versions of the same protocol. 
Not only is this very costly in development time, the core logic is rewritten and re-deployed time and time again, and with each new deployment, new bugs can be introduced.

Being able to build on top of battle tested code would benefit smart contract developers, protocols, users and the overall ecosystem.

\subsubsection{Fragmented and isolated collateral}
\label{subsubsec:fragmented-collateral}

The average DeFi-user has collateralized assets fragmented and isolated across a multitude of protocols (do we have any numbers on this?), and many quality assets are sitting idle.
While isolated margin positions have there benefits (they isolate risks), they are not capital efficient.
Having the ability to use a portfolio of assets as collateral improves the capital efficiency for a number of reasons:
\begin{itemize}
    \item The volatility of a portfolio of assets is always equal to, or lower than the weighted volatility of the individual assets.
    Or put different, assets losing in value can be compensated with assets increasing in value.
    \item Having less positions, with lower volatility, reduces the nu
    \item Depending on the correlation between collateralized assets, the Creditor might use less strict margin requirement.
    \item Debtors can use negatively correlated assets to hedge positions and limit liquidation risks.
\end{itemize}

Having a global shared state across applications and assets is often cited as one of the key advantages of blockchain over centralised financial infrastructures\cite{schar2021decentralized}.
This should open up a whole new solution space for Debtors to manage collateralized positions, 
and might even enable a single Debtor to share its margin between non-trusting Creditors without intermediaries.
Hence it is quite ironic that traditional brokers and clearing houses offer more advanced capabilities for cross- and portfolio-margined positions, compared to the state-of-the-art DeFi protocol.

Again there are again multiple underlying root causes why collateral in DeFi is still fragmented and under-utilized:
\begin{itemize}
    \item As mentioned in Section \ref{subsubsec:hacks-and-exploits}, there is no shared collateral management layer, each protocol with collateral has their own non-standardized implementation.
    \item Protocols are build around specific assets types (eg. lending protocols for simple ERC20 or for AMM LPs, or for NFTs). 
    Different asset types cannot be used within the same protocol to back a single position and emergence of new primitives/token standards requires migrations/new protocols.
    \item Blockchain is still a young technology, core non-financial infrastructure required for on-chain portfolio management (think Account Abstraction, intents or cross-chain messaging) is only recently developed. 
\end{itemize}

Fragmentation of assets not only results in capital inefficiencies, it also contributes to a challenging user experience, more on that in the next Section (\ref{subsubsec:end-user-complexity}).

\subsubsection{End user complexity}
\label{subsubsec:end-user-complexity}

End-user do not "want" to manage collateral, they want to optimize portfolio's to achieve a certain objective (e.g. increase yield, delta hedging, diversify exposure to different protocols...).
Collateral management is a means to an end, it is not an activity most users enjoy doing.

End-user adoption will only increase outside of a niche bubble of tech enthusiasts if the technical complexities around collateral management are be abstracted away.
Important note, abstracting away technical complexities should never come at a cost of hiding risks or relying on centralised and custodial solutions (as is to often the case).

As mentioned in the previous Section \ref{subsubsec:fragmented-collateral}, the bad user experience is partly due to the fragmentation of assets and positions.
Rebalancing portfolio's often require multiple transactions per asset and per position.
Today the user has to execute multiple sequential transactions per asset, instead a single transaction that rebalances the complete portfolio.
Let's take as example a user that wants to earn yield from his collateralized assets and get exposure to Liquid Staking Tokens (LSTs).
Since LST protocols introduce an additional layer of risk, he wants to diversify risks over 5 different LST service providers.
Assuming our user has wrapped Ethereum, he would need to do 10 transactions via multiple platforms (5 approvals and 5 swaps or 5 approvals and 5 staking actions).
Rebalancing said portfolio, or depositing it as collateral, would require another 5 to 10 transactions.
Not only is this time consuming, it also introduces additional operational risks, to name a few:
 \begin{itemize}
    \item Herstatt risk (settlement risk) due to changing markets before each leg of the action is settled.
    \item Increased risk of manual mistakes (fat fingers, bad slippage settings).
    \item Increased risk of falling victim phishing attacks on one of the platforms.. 
\end{itemize}

A second abstraction is to let users define what they want, and provide them with the information (the calldata) how to do it.
Since most collateral management actions require interactions with multiple assets/protocols,
this abstraction can only be achieved after portfolio's as a whole can be managed with a single atomic transaction.

At the time of writing this paper, the required infrastructure to enable both abstractions, is heavily debated within the broader ecosystem and multiple proposals are launched to standardize the infrastructure.
Most notable are the proposals for Account Abstraction (e.g. EIP-3074 and EIP-4337) and for Intent based architectures (e.g. EIP-7521)

Both abstractions are already successfully applied (albeit in a somewhat non-standardized form) within DeFi by Decentralised Exchanges, NFT-marketplaces and their aggregators.

Our philosophy is that users should express what they want to do, not how they do it.

\subsection{New Architectures}
During the last months many solutions were proposed by different protocols and stakeholders on how to build a more robust decentralised financial system.
Notable recent papers are from Uniswap V4\cite{adams2023uniswap}, Morpho\cite{gontier2023morpho} and Euler\cite{euler2023protocols}.
All mention similar concepts as the distinction between product and protocol, modularization of the protocol, oracle agnostic implementations and explicitly separating the logic of the protocol in different layers,
where layers with different complexity evolve at different speeds.

Especially the last concept, to separate the logic in different layers, resonates with the long term vision of the Arcadia Finance team.
Different Logic layers of the financial stack, with different complexities should have different life-cycles and innovation timelines.
With the lowest most core component the slowest moving, while the upper user facing layers must evolve and adapt quick in response ever changing markets.

A good practical example is uniswap V4, where Uniswap Labs implements the underlying core mathematical logic for CLAMMs (Concentrated Liquidity Automated Market Makers).
Other teams can build in a permissionless manner on top of the Uniswap V4 protocol and develop feature rich and fast innovating new DEXs, without the need and risks of redundantly implementing the complex maths.

Also over-collateralized protocols would benefit from a similar architecture.
A shared standardized and permissionless layer with battle tested logic for collateral management, on top of which Creditors build their application specific financial contracts.

Bringing back pristine focus to collateralization is exactly what we are doing with the Arcadia Financial Technology Stack.
We believe that taking a collateral management first approach results in overall better DeFi protocols.


\section{The Arcadia Financial Technology Stack}

the Arcadia Financial Technology Stack is implemented for the Ethereum Virtual Machine (EVM) and consists out of three protocols:
\begin{itemize}
\item The Arcadia Registry.
\item Arcadia Accounts.
\item A Creditor contract.
\end{itemize}

General overview of the protocols, design requirements, permissions.
- General philosophy: creditor is the maker, debtor is the taker.
- Append only, no overwriting 

\lipsum[2]

\section{Arcadia Registry}
\label{sec:arcadia-registry}

More details on requirements + technical implementation.

The registry is a modular and append-only registry for pricing logic and risk data of on-chain assets.
Given the importance and nature of collateral, infrastructure to price collateral is of key importance that collateral should be:
- Should be safe
- Should be gas efficient
- Should be liquid
- Should be Should be resistant to price manipulation via flash loans
- The size of the liability it backs, should be small compared to the liquid market size.

\subsection{Many assets to one pricing}
Creditor sets "routing".

\subsection{Modular pricing logic}
In the Arcadia protocol, there is not a single contract responsible for pricing assets, instead we work with a single Registry and multiple Pricing Modules.
Each Pricing Module is a separate smart contract with the pricing logic for specific Oracle implementations or Asset types.
The Registry keeps mappings per Creditor of which Module to use to price a certain asset.

It is the Creditor itself that sets these mappings, the Creditor decides which Modules to use for each asset they allow their Debtors to use as collateral.
Different Creditors can use different Modules to price the same asset, or a Creditor can choose to not allow certain assets at all.

Modules can only be appended to the protocol, not removed or overwritten.
This ensures that Pricing logic is immutable, but it still gives flexibility to add new assets, or implement more efficient Pricing logic.

There are two categories of modules: Oracle Modules and Asset Modules.

\subsubsection{Oracle Modules}
Each different oracle implementation (e.g. Chainlink oracles, Pyth oracles, Uniswap TWAPs...) should have its own Oracle Module.
The Oracle Module has two main functionalities.
Firstly, it returns the oracle-rate in a standardized format.
And secondly it checks if the oracles perform as intended.
If not, non-functioning oracles are decommissioned.

Each oracle has a quote-asset and a base-asset, the oracle-rate $R_{oracle}$ reflects how much units of the quote-asset are required to buy 1 unit of the base-asset.
The precision $S_{oracle}$ of oracles is variable and can even be a binary fixed-point number.
Since all pricing logic within the Arcadia Protocol uses fixed-point numbers with 18 decimals precision,
a correction from the oracle precision to a precision of 18 decimals is required for the standardized oracle rate $R_{ba\rightarrow qa}$, returned by the Oracle Module.
\begin{equation}
    \label{eq:oracle-module}
    R_{ba\rightarrow qa} = R_{oracle} \frac{10^{18}}{S_{oracle}}
\end{equation}

\subsubsection{Asset Modules}
Just as each oracle implementation should have its own oracle Module, each asset type should have it's own Asset Module.
Each Asset Module has three main functionalities:
\begin{itemize}
    \item Implement the pricing logic to calculate the USD value of an asset. 
    \item Process Deposits and Withdrawals.
    \item Manage Asset specific risk parameters.
\end{itemize}

Asset Modules can be further divided into two distinct groups: Primary Asset Modules and Derived Asset Modules.

\subsubsection{Primary Asset Modules}
Primary assets are defined as assets that are not composed of other assets.
Primary Assets must be priced using one or more on-chain oracles, for which the values are fetched via the Oracle Modules.

The Registry will enforce that each asset of a Primary asset is priced in USD with 18 decimals precision.
Since the asset amount, $A_{asset}$, can have a variable precision $S_{asset}$, a correction is again applied to bring the usd-value of the asset $V_{asset}^{USD}$ to 18 decimals precision.

Take as example 
\subsubsection{Derived Asset Modules}

\subsection{Risk Mitigation}
It is impossible to guarantee that every module will always be bug free.
A protocol where new pricing modules can be appended, but that relies on the assumption that all bugs will be caught before deployment of each module, is bound to fail.
A bug in a single module should not lead to the end of the entire protocol and to all protocols build on top of it.

The probability that bugs might occur in a module at some point should be taken into account in the protocol design.
This consists of three steps: impact mitigation, impact isolation and module replacement.

Risk parameters should be granular enough to be able to capture the smart contract risk for novel modules.
Different Creditors should be able to decide when they enable their debtors to use assets priced by a certain module as collateral.
And when they decide to enable the module, set upper boundaries to the overall exposure to said modules.
As such, if one module is faulty, it should be possible to mitigate it's impact.

After a bug is identified



\section{Arcadia Accounts}
\label{sec:arcadia-accounts}

\subsection{Multi-collateral}
By allowing multiple types of assets to be deposited within a single Account, users, institutions and protocols can build diversified portfolio's.
The list of asset(types) that can be deposited in Arcadia Accounts is continuously being expanded. Our goal is for users to unlock capital of any quality, price-able DeFi primitive.

Each Creditor can determine themselves which asset(types) are to be allowed as collateral in Accounts of their Debtors.

\subsection{Composable}
Arcadia Accounts themselves are according to the ERC-721 standard and can thus be represented as a single asset.
This means that Accounts are fully composable with existing infrastructure and are straightforward to integrate.
Other benefits are that Accounts can be tracked and monitored trough existing DeFi dashboards and it opens up use-cases where Accounts can be sold in their entirety (both assets and liabilities) on existing NFT marketplaces.

\subsection{Token Standard Agnostic}

\subsection{Margin Accounts}

\subsection{Account Abstraction ready}

\subsection{Flash Actions}
Flash Actions (or optimistic Actions or flash accounting) expands on the concept of flashloans, and are only possible thanks to a unique property of smart contracts: atomicity\cite{xie2022towards}.
Just as with flashloans, each step of the Flash Action must be successful or the transaction as a whole fails.
The final health check of the Account (assets can cover all liabilities) is done at the very end of the transaction.
This allows the Account Owner to temporary bring the Account in an undercollateralized (or even non collateralized state) without the risk on bad debt for any Creditor.
Since if the Account is not brought back into a healthy state during the transaction, the final health check fails and thanks to atomicity the whole transaction fails.

This gives Account Owners unprecedented flexibility to manage assets and liabilities.
In a fully permissionless way they can chain the following together (= do a flash action):
\begin{itemize}
    \item A margin Account can be opened for a new Creditor, if the new Creditor is approved by the Account Owner.
    \item The Creditor can execute arbitrary logic (e.g. give a flashloan).
    \item The Account owner can optimistically withdraw assets from the Account.
    \item The Account owner can transfer assets from his own wallet to the Account or external logic.
    \item The Account owner can execute external logic, where he uses the assets (loaned, withdrawn or transferred) to interact with multiple DeFi protocols to swap, stake, claim...
    \item The Account owner can deposit recipient tokens back into the Account.
\end{itemize}
The only requirement is that the Account is in a healthy state at the very end of the flash action.

Flash Actions are a very powerful tool that has no equal in traditional finance.
We will list a few examples how flash actions from an Arcadia Account can be used out of the box (the list is far from exhaustive).
And keep in mind that all of these actions can be done, even if the assets are used to secure liabilities.

\begin{itemize}
    \item Rebalance whole portfolio's, swapping a portfolio of n different assets directly to a new portfolio of m different assets.
    \item Refinance liabilities (change to a different Creditors), without the need to sell any collateral.
    \item Stake or provide liquidity on approved DeFi protocols.
    External protocols used in this context will need to provide a receipt token which must be allowed as collateral as well within the Arcadia protocol.
    Examples can be providing liquidity on Aave (receiving approved aTokens), depositing assets on Yearn (receiving approved yTokens), …
    \item Change ranges for Uniswap V3 LP. Contrary to Uni V2 and similar AMMs, Uni V3 positions are meant to be more actively managed in terms of liquidity ranges.
    Users that deposit Uni V3 positions in their Arcadia Accounts will have the ability to change those liquidity ranges without having to withdraw their tokens first.
    \item Claim airdrops that depend on address-owned tokens. Arcadia Accounts will feature “flash withdrawals”.
    This feature can be used by the Account owner to claim airdrops using assets under collateral within their Arcadia Account, without having to close DeFi positions to withdraw their tokens first.
    \item ...
\end{itemize}

\subsection{Intent ready}

\subsection{Automated Asset Management}

\subsection{Upgradeability}
Each deployed Account is linked to a certain Account logic contract.
Account logic will include the features Account owners can benefit from, for example, flash withdrawals, active collateral management, authorization delegation, … 
The Account logic is upgradable, but the user has full control if and when they want to upgrade to new logic.
Should a user wish to use features newly introduced in an upcoming Account version, it will be up to them to upgrade the linked Account logic.
Users will not have to migrate assets or close DeFi positions when doing so. 

Creditors can determine which Account logic versions are allowed to be used by their Debtors.
As such, highly-customized Account logic can be implemented for protocol-specific versions should this be required. 
The Arcadia protocol can in no way change the version of a user-owned deployed Account, or change any functionality in existing Account logic.

\section{Arcadia Creditor}
\label{sec:arcadia-creditor}

\subsection{Generalised Creditor}

\subsection{Composable}

\subsection{Risk Management}

\subsection{Accounting Liabilities}

\subsection{Liquidations}

\lipsum[5]

\subsection{Arcadia Lending Pools}

\lipsum[6]

\begin{figure}
    \label{fig:arcadia-logo}
    \centering
    \includegraphics[width=0.3\textwidth]{images/Logo-Arcadia.png}
    \caption{Example of an image.}
    \Description{Example of an image.}
  \end{figure}

\section{Summary}
\lipsum[7]

\bibliographystyle{ACM-Reference-Format}
\bibliography{main}

\section*{Disclaimer}
\lipsum[8]

\end{document}

\endinput
