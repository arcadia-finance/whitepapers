\documentclass[sigconf,nonacm]{acmart}

%% packages
\usepackage{graphicx}
\usepackage{lipsum}

\title{Arcadia Finance [Draft]}
\subtitle{November 2023}
\date{November 2023}

\author{Thomas Smets}
\affiliation{
    \institution{Arcadia Finance}
    \city{Brussels}
    \country{Belgium}
}
\email{thomas@arcadia.finance}

\renewcommand{\shortauthors}{Smets et al.}

%% Add whitespace
\begin{teaserfigure}
    \vspace{1em}
    \Description{Just used to add a whitespace over the length of the document.}
\end{teaserfigure}

\begin{document}

\begin{abstract}
    \lipsum[1]
\end{abstract}

\keywords{DeFi, EVM, collateral}

\maketitle

\section{Introduction} 
\label{sec:introduction}

This paper will explain the rationale why we are building the Arcadia Finance Technology Stack, and go over its technical implementation.

The Arcadia Financial Technology Stack is a set of on-chain, inter-linked financial protocols, centered around collateral management.
They facilitate non-trusting peers or contracts to close financial contracts without the need of intermediaries.

It consists out of three protocols, each with distinct responsibilities and permissions:
\begin{itemize}
\item The Arcadia Registry.
\item Arcadia Accounts.
\item A Creditor contract.
\end{itemize}

\subsection{Terminology}
We will often use the terms Debtors and Creditors throughout this paper, our definition is broader as the commonly accepted context within traditional debt arrangements.
We use the term Debtor for any entity with financial liabilities on the balance sheet (borrowers, option writers, payers of cash-settled futures contract or swaps...). 
While the Creditor refers to the entity to whom the liability is owed.

\subsection{Collateral Management}
Withing the financial system, Collateral serves a primary function in mitigating the counterparty risks borne by Creditors, in case Debtors would fail to fulfill their financial liabilities.

\subsubsection{History}
Evidence of collateralized loans, also known as secured loans, can be traced back to at least 400 BCE in Ancient Greece  \cite{millett2002lending}.
In these early transactions, borrowers provided tangible assets as collateral to secure loans, laying the foundation for the concept of collateralization.

Over the past decades we have witnessed significant advancements in collateral management practices. 
On the one hand, with the rise of complex financial instruments and the globalization of markets, the role of collateral has expanded beyond traditional lending to secure a wide range of financial instruments,
including derivatives trading, margin accounts, and other sophisticated financial contracts.

On the other hand, the digitalisation of the financial system greatly improved the efficiency of the exchange and settlement of collateral.
Reporting, margining and reconciliation processes could be executed on a daily basis instead of on a weekly or even monthly basis \cite{simmons2019collateral}.

The emergence of blockchain technology and decentralized finance (DeFi) led to a new wave of innovation for collateral management.
Blockchain has some excellent properties with regards to collateral management:
\begin{itemize}
    \item With it's immutable and transparent ledger, collateral can be audited 24/7.
    \item Smart contracts enforce margin requirements real time on a 24/7 continuous basis.
    \item The atomic execution of transactions avoid expensive reconciliation processes.
    \item The atomic execution also enables optimistic execution of transactions (e.g. flash loans) which can greatly improve processes such as refinancing loans or even enable completely new financial use-cases.
    \item The permissionless nature of blockchain and Dapps (Decentralised Applications), combined with a shared state between all those applications, make different financial applications and assets composable and inter-operable by default.
\end{itemize}

It is no coincidence that some of the first blockchain applications with product market fit, were secured secured lending protocols.
These early pioneers like Maker\cite{team2017dai}, Aave\cite{thornburg2020aave} or Compound\cite{leshner2019compound} relied on collateralization as the only way to mitigate counter party risks and work with over-collateralized loans.
As such they enable peer-to-peer or even peer-to-contract loans, without the need of any trusted intermediaries.

\subsubsection{Shortcomings}
Collateralization, both in the traditional financial world and in DeFi, is not without its problems.
Recent crisis in both industries industries highlight the need for better collateral management infrastructure.

Bad collateral management, rehypothecation of collateral, opaque accounting practices of collateral, and outright fraud with collateralized assets, are all cited as root causes\cite{hellwig2008causes} behind the 2008 financial crisis.
Following the crisis, a variety of major regulations were introduced (e.g. EMIR in the European Union, or Dodd-Frank in the USA).
While these packages were successful in stabilizing the financial system, they have clear centralization tendencies\cite{gregory2014central}.
The major beneficiaries of said regulations are the major financial institutions, mandatory intermediaries for central clearing and the regulatory bodies themselves.
This might paradoxically cement even greater systemic risks into the system.

In 2021 and 2022 DeFi experienced it's own financial crisis.
Where in a few months many of the ecosystem's key (albeit mostly centralised) players collapsed, removing all liquid with them.
Again bad collateral management, rehypothecation of collateral, opaque protocol mechanisms, and outright fraud with collateralized assets are at the root of the problem.

A recurring trend with protocols that imploded in 2021 and 2022 is that the lack of collateralization was obfuscated behind complex protocol designs and narratives. 
Some of the notorious examples are:
\begin{itemize}
    \item Synthetic stable-tokens such as Gaia-USDf, Iron-Titan and Luna-Terra.
    \item Olympus and its forks relying on the (3,3) model.
    \item Protocols where the only utility of a token is to receive more of the token itself (yield farms, reflective tokens…).
    \item Unsecured loans by market makers and CeFi players.
\end{itemize}

\subsubsection{Shortcomings 2}
Next to the aforementioned under-collateralized protocols, also the over-collateralized protocols are not without problems.

A major problems in DeFi today is the number of hacks and exploits, estimates of total funds lost in 2023 range from \$400M to \$1B.

During the last months many solutions were proposed by different protocols and stakeholders on how to build a more robust decentralised financial system.
Notable papers are for instance from Morpho\cite{gontier2023morpho} and Euler\cite{euler2023protocols}.
Both mention similar concepts as the distinction between product and protocol, modularization of the protocol, oracle agnostic implementations and explicitly separating the protocol in different layers.

Many concepts resonate with the Arcadia team, but in others we take a different approach.
Both focus on isolated lending primitives, while there

A second problem we identified is that the logic to price and manage collateral is always DeFi protocols itself:
- Redundant logic
- Core pricing logic is rewritten and re-deployed time and time again, with each new deployment, new bugs can be introduced.
Relying on Battle tested code both benefits smart contract developers, users and the overall ecosystem

Given the importance and nature of collateral, infrastructure to price collateral is of key importance that it should be safe:
- Should be safe
- Should be gas efficient
- Should be liquid
- Should be Should be resistant to price manipulation via flash loans
- The size of the liability it backs, should be small compared to the liquid market size.

\subsubsection{The way forward}

Different layers of the financial stack should have different life-cycles and innovation timelines.
With the lowest most core component the slowest moving, while the user facing upper layers must evolve and adapt quick in response ever faster changing markets.

Bringing back pristine focus to collateralization is exactly what we are doing with Arcadia Finance.
We believe that taking a collateral management first approach results in overall better DeFi protocols. 

Arcadia strives to enable higher capital efficiency for users, transparent risk assessment, faster go-to-market time for new protocols without compromising on self-custody and decentralisation.

\section{Arcadia Registry}
\label{sec:arcadia-registry}
The registry is a modular and append-only registry for pricing logic and risk data of on-chain assets.
to


\section{Section 1} 
\label{sec:section1}

\lipsum[3]

\subsection{Subsection 1} 
\label{subsec:subsection1}

\lipsum[4]

\begin{itemize}
\item test
\item test
\item test
\item test
\end{itemize}


\subsection{Subsection 2} 
\label{subsec:subsection2}

\lipsum[5]

\begin{figure}
    \label{fig:arcadia-logo}
    \centering
    \includegraphics[width=0.3\textwidth]{images/Logo-Arcadia.png}
    \caption{Example of an image.}
    \Description{Example of an image.}
  \end{figure}

\section{Section 2} 
\label{sec:section2}

\lipsum[6]


\section{Summary}
\lipsum[7]

\bibliographystyle{ACM-Reference-Format}
\bibliography{main}

\section*{Disclaimer}
\lipsum[8]

\end{document}

\endinput
